\chapter{Single Transferable Vote (STV) Instructions}

\appnpara These rules shall be followed for the counting of votes in Joint Elections to the Executive Committee. In this appendix, terms shall be interpreted as in section~\ref{STV:Defs}

\section{First Stage}
\appnpara The returning officer shall sort the valid ballot papers into parcels according to the candidates for whom first preference votes are given.
\appnpara The returning officer shall then
\begin{enumerate}
    \item count the number of ballot papers in each parcel;
    \item credit the candidate receiving the first preference vote with one vote for each ballot paper; and
    \item record those numbers.
\end{enumerate}
\appnpara The returning officer shall also ascertain and record the total number of valid ballot papers.

\section{The quota}

\appnpara The returning officer shall divide the total number of valid ballot papers for the electoral ward by a number exceeding by one the number of councillors to be elected at the election for that electoral ward.
\appnpara The result of the division under paragraph (1) (ignoring any decimal places), increased by one, is the number of votes needed to secure the return of a candidate as a councillor (in these rules referred to as the "quota").

\section{Election of Candidates}

\appnpara Where, at any stage of the count, the number of votes for a candidate equals or exceeds the quota, the candidate is deemed to be elected.

\section{Transfer of ballot papers}

\appnpara Where, at the end of any stage of the count, the number of votes credited to any candidate exceeds the quota and, subject to rules 49 and 52, one or more vacancies remain to be filled, the returning officer shall sort the ballot papers received by that candidate into further parcels so that they are grouped
\begin{enumerate}
    \item according to the next available preference given on those papers; and
    \item where no such preference is given, as a parcel of non-transferable papers.
\end{enumerate}

\appnpara The returning officer shall, in accordance with this rule and rule 49, transfer each parcel of ballot papers referred to in paragraph (1)(a) to the continuing candidate for whom the next available preference is given on those papers and shall credit such continuing candidates with an additional number of votes calculated in accordance with paragraph (3).
\appnpara \label{STV:4.3} The vote on each ballot paper transferred under paragraph (2) shall have a value ("the transfer value") calculated as follows: 
$\frac{A}{B}$, where $A =$ the value which is calculated by multiplying the surplus of the transferring candidate by the value of the ballot paper when received by that candidate; and $B =$ the total number of votes credited to that candidate.

The calculation is made to five decimal places (any remainder being ignored).

\appnpara For the purposes of paragraph~\ref{STV:4.3},
\begin{enumerate}
    \item ``transferring candidate'' means the candidate from whom the ballot paper is being transferred; and
    \item``the value of the ballot paper'' means
    \begin{enumerate}
        \item for a ballot paper on which a first preference vote is given for the transferring candidate, one; and
        \item in all other cases, the transfer value of the ballot paper when received by the transferring candidate.
    \end{enumerate}
\end{enumerate}


\section{Transfer of ballot papers — supplementary provisions}

\appnpara If, at the end of any stage of the count, the number of votes credited to two or more candidates exceeds the quota the returning officer shall—
\begin{enumerate}
    \item first sort the ballot papers of the candidate with the highest surplus; and
    \item then transfer the transferable papers of that candidate.
\end{enumerate}


\appnpara If the surpluses determined in respect of two or more candidates are equal, the transferable papers of the candidate who had the highest number of votes at the end of the most recent preceding stage at which they had unequal numbers of votes shall be transferred first.

\appnpara If the numbers of votes credited to two or more candidates were equal at all stages of the count, the returning officer shall decide, by lot, which candidate's transferable papers are to be transferred first.

\section{Exclusion of candidates}

\appnpara If, one or more vacancies remain to be filled and—
\begin{enumerate}
    \item the returning officer has transferred all ballot papers which are required by rule 48 or this rule to be transferred; or
    \item there are no ballot papers to be transferred under rule 48 or this rule, the returning officer shall exclude from the election at that stage the candidate with the then lowest number of votes.
    
\end{enumerate}

\appnpara The returning officer shall sort the ballot papers for the candidate excluded under paragraph (1) into parcels so that they are grouped—
\begin{enumerate}
    \item according to the next available preference given on those papers; and
    \item where no such preference is given, as a parcel of non-transferable papers.
\end{enumerate}

\appnpara The returning officer shall, in accordance with this article, transfer each parcel of ballot papers referred to in paragraph (2)(a) to the continuing candidate for whom the next available preference is given on those papers and shall credit such continuing candidates with an additional number of votes calculated in accordance with paragraph (4).

\appnpara The vote on each ballot paper transferred under paragraph (3) shall have a transfer value of one unless the vote was transferred to the excluded candidate in which case it shall have the same transfer value as when transferred to the candidate excluded under paragraph (1).

\appnpara This rule is subject to rule 52.

\section{Exclusion of candidates — supplementary provisions}

\appnpara If, when a candidate has to be excluded under rule 50—
\begin{enumerate}
    \item two or more candidates each have the same number of votes; and
    \item no other candidate has fewer votes,
\end{enumerate}
paragraph (2) applies.

\appnpara Where this paragraph applies—
\begin{enumerate}
\DrawEnumitemLabel
    \item regard shall be had to the total number of votes credited to those candidates at the end of the most recently preceding stage of the count at which they had an unequal number of votes and the candidate with the lowest number of votes at that stage shall be excluded; and
    \item where the number of votes credited to those candidates was equal at all stages, the returning officer shall decide, by lot, which of those candidates is to be excluded.
\end{enumerate}

\section{Filling of last vacancies}

\appnpara Where the number of continuing candidates is equal to the number of vacancies remaining unfilled, the continuing candidates are deemed to be elected.

\appnpara Where the last vacancies can be filled under this rule, no further transfer shall be made.

\section{Definitions} \label{STV:Defs}
\begin{description}
\item[continuing candidate] any candidate not deemed to be elected and not excluded from the list of candidates under rule 50;

\item[next available preference] a preference which is the second or, as the case may be, subsequent preference in consecutive order for a continuing candidate (any preferences for any candidate who is deemed to be elected or is excluded from the list of candidates under rule 50 being ignored);
\item[non-transferable paper] a ballot paper on which there is no next available preference;
\item[quota] has the meaning given in rule 46;
\item[returning officer] the Returning Officer, or any Ordinary Member validly chosen to assist with the Count.

"spoilt ballot paper" has the meaning given in rule 36;

\item[stage of the count] means
\begin{enumerate}
    \item the determination of the number of votes for each candidate as first preference;
    \item the transfer of transferable papers from a candidate deemed to be elected who has a surplus; or
    \item the exclusion of a candidate at any given time;
\end{enumerate}

\item[surplus] the number of votes, if any, by which the total number of votes credited to a candidate deemed to be elected exceeds the quota;

\item[transferable paper] a ballot paper on which a next available preference is given;

\item[transfer value] the value of a vote on a ballot paper calculated in accordance with rule 48;

\end{description}