\chapter{Single Transferable Vote (STV) Instructions}

\textbf{Detailed STV Instructions ( ERS97)}

These instructions are taken from How to Conduct an Election by the Single Transferable Vote (Third Edition 1997) which may be found in its entirety at: \\ \\ http://www.crosenstiel.webspace.virginmedia.com/stvrules/index.html \\ \\
The sections shown below (4-6) are to be used for the conduct of JCR Elections by Single Transferable Vote.  The other sections are informative and advisory and are either unnecessary or superseded by instructions in S.O.4. \\
This variation of STV is generally referred to as ERS97. The colour coding suggested is optional. \\

\section{General description of the count}
\appnpara The count is divided into a number of stages. At the first stage the voting papers are counted to determine the total vote. They are then sorted according to their first preferences, and any papers which are invalid are removed. The total number of valid votes is then found and the quota calculated. Any candidates who have at least a quota of first preference votes are deemed elected at this stage.
\appnpara Each subsequent stage of the count is concerned either with the transfer of surplus votes of a candidate whose vote exceeds the quota, or with the exclusion of one or more candidates with the fewest votes.
\appnpara This procedure continues until either sufficient candidates have reached the quota to fill all the seats, or there is the same number of candidates left as unfilled seats.
\appnpara These rules refer to the various forms published by the Electoral Reform Society. The use of these forms is optional, but where they are used, the various options should be made easier, particularly for those not experienced in conducting STV counts.
\appnpara In the rules below, words in \textbf{bold type} indicate that there is a definition in the glossary (section 6).

\section{Detailed instructions for the count}

In a public election, it is necessary to include certain formalities, such as unsealing and opening the ballot boxes at the start, checking the number of papers in each and ascertaining that the candidates and their agents are content at the conclusion of each stage. For simplicity these have been omitted from these instructions.

\subsection{First stage}
\appnpara Count all the voting papers to determine the total number of votes cast.
\appnpara Sort the voting papers into \textbf{first preferences}, setting aside any \textbf{invalid papers}. Count the number of invalid papers, and subtract this from the total vote to get the \textbf{total valid vote}.
\appnpara Check the sorting, and count the papers for each candidate into bundles, inserting a \textbf{counting slip }(green) in each bundle marked with the name of the candidate, the number of papers, and ``first stage". For very small elections, the use of counting slips may be dispensed with.
\appnpara Check the counting. Enter on each candidate’s \textbf{vote record form} (yellow) the total number of\textbf{ first preference votes}.
\appnpara Copy the \textbf{candidates' votes} from the vote record forms onto a\textbf{ result sheet} (white), and check that their total is the same as the total valid vote.
\appnpara Calculate the \textbf{quota} by dividing the total valid vote by one more than the number of places to be filled. Take the division to two decimal places. If the result is exact that is the quota. Otherwise ignore the remainder, and add 0.01.
\appnpara Considering each candidate in turn in descending order of their votes,\textbf{ deem elected} any candidate whose vote equals or exceeds:

\begin{enumerate}
    \item the quota, or
    \item (on very rare occasions, where this is less than the quota), \textbf{the total active vote}, divided by one more than the number of places not yet filled, up to the number of places to be filled, subject to paragraph 5.6.2.
\end{enumerate}
\appnpara That completes the first stage of the count. Now proceed to section 5.2 below.

\subsection{Subsequent stages}
Each subsequent stage will involve either the distribution of a surplus, or, if there is no surplus to distribute, the exclusion of one or more candidates.
 If one or more candidates have surpluses, the largest of these should now be transferred. However the transfer of a surplus or surpluses is deferred and reconsidered at the next stage, if the total of such surpluses does not exceed either:
(a)    The difference between the votes of the two candidates who have the fewest votes, or
(b)    The difference between the total of the votes of two or more candidates with the fewest votes who could be excluded under rule 5.2.5, and the vote of the candidate next above.
If one or more candidates have surpluses which have not been deferred, transfer the largest surplus. If the surpluses of two or more candidates are equal, and they have the largest surplus, transfer the surplus of the candidate who had the greatest vote at the first stage or at the earliest point in the count, after the transfer of a batch of papers, where they had unequal votes. If the votes of such candidates have been equal at all such points, the Returning Officer shall decide which surplus to transfer by lot.
The transfer of a surplus constitutes a stage in the count. Details of how to do this are in section 5.3. If, after completing the transfer, there are still any untransferred surpluses, and not all the places have been filled, proceed as in paragraph 5.2.2
If, after all surpluses have been transferred or deferred, one or more places remain to be filled, the candidate or candidates with the fewest votes must be excluded. Exclude as many candidates together as possible, provided that:
    (a)    Sufficient candidates remain to fill all the remaining places
    (b)    The total votes of these candidates, together with the total of any deferred surpluses, does not exceed the vote of the candidate next above.
If the votes of two or more candidates are equal, and those candidates have the fewest votes, exclude the candidate who had the fewest votes at the first stage or at the earliest point in the count, after the transfer of a batch of papers, where they had unequal votes. If the votes of such candidates have been equal at all such points the Returning Officer shall decide which candidate to exclude by lot.
Details of how to exclude a candidate are given in section 5.4.
Exclusion of one or more candidates constitutes a stage in the count. If, after completing this, there are any surpluses to transfer, and not all the places have been filled, proceed as in paragraph 5.2.2. Otherwise proceed to exclude further candidates as in paragraph 5.2.5.
Transfer of a surplus
If a surplus arises at the first stage, select for examination all the papers which the candidate has received.
If a surplus arises at a later stage, because of the transfer of another surplus or the exclusion of a candidate or candidates, select only the last received batch of papers, which gave rise to the surplus.
Examine the selected voting papers and sort them into their next available preferences for continuing candidates. Set aside as non-transferable papers any on which no next available preference is expressed.
Check the sorting, count and bundle the papers now being transferred to each candidate, also any non-transferable papers. Insert a counting slip in each bundle marked with the stage number, the name of the candidate to whom the papers are being transferred, and the number of papers in the bundle.
Count the number of transferable papers and enter the number for each candidate on the vote record forms.
Prepare a surplus form (pink). Copy the number of papers for each candidate from the vote record forms to the surplus form, and check the total.
Calculate the total value of the transferable papers. If this exceeds the surplus, determine the transfer value of each paper by dividing the surplus by the number of transferable papers, to two decimal places, ignoring any remainder. If the total value does not exceed the surplus, the transfer value of each paper is its present value.
Calculate the value to be credited to each candidate by multiplying the transfer value by the number of papers, check the totals, and enter these on the surplus form.
Copy the values to be credited, and the non-transferable difference arising from the neglected remainder, from the surplus form to the vote record forms and to the result sheet.
Add these values to the previous votes for each candidate, and add the non-transferable difference to the previous total of non-transferable votes, entering the figures onto the vote record forms and the result sheet.
Add up the new total number of votes on the result sheet, and check that this still equals the original total valid vote.
Complete the counting slips with the transfer value of each paper, and place the bundles of voting papers for each candidate with those previously received. In a small election, where counting slips are not being used, each ballot paper should be marked with its transfer value.
Considering each continuing candidate in turn in descending order of their votes, deem elected any candidate whose vote now equals or exceeds
Transfer of the votes of excluded candidates
Take together all the bundles of papers which are currently credited to the candidate or candidates to be excluded, and arrange them in batches in descending order of transfer value. Check that the number and total value of the papers in each batch agrees with the numbers on the vote record forms and the result sheet. Prepare an exclusion form (blue).
First, take the batch of papers with the highest transfer value. Sort them according to the next available preferences for continuing candidates, and set aside as non-transferable papers any on which no next available preference is expressed.
Check the sorting, count and bundle the papers for each candidate and any non-transferable papers. Insert a counting slip in each bundle stating the stage, the name of the candidate to whom the papers are being transferred, the number of papers, and the transfer value of each paper. If counting slips are not being used, the transfer value should be marked on each paper.
Check the counting and enter the number of papers for each candidate and the number of non-transferable papers on the vote record forms.
Copy the number of papers to be transferred to each candidate and the number of non-transferable papers, from the vote record forms onto a column of the exclusion form, and check the total.
Determine the total value of the papers for each candidate and that of the non-transferable papers and check the total.
Copy the total values for each candidate from the exclusion form to the vote record forms, and place the bundles of voting papers for each candidate with those previously received.
If any papers have become non-transferable before any candidate has been deemed elected, recalculate the quota as in paragraph 5.1.6, ignoring the non-transferable vote.
Considering each continuing candidate in turn in descending order of their votes, deem elected any candidate whose vote now equals or exceeds
    (a)    the quota, or
    (b)    the total active vote, divided by one more than the number of places not yet filled,
    up to the number of places remaining to be filled, subject to paragraph 5.6.2.
Ensure that no further papers are given to candidates who are no longer continuing candidates because they have been deemed to be elected after transferring a batch of papers.
As in paragraph 5.4.2 and subsequently, sort and transfer each batch of papers in turn in descending order of transfer value, complete a column of the exclusion form for each batch, and deem candidates to be elected as appropriate.
After all the batches of papers have been transferred, the right hand (totals) column on the exclusion form should be completed and these totals checked against the vote record form(s) of the excluded candidate(s).
Copy the total values to be credited from the exclusion form to the vote record forms and to the result sheet, and add these to the previous totals for each candidate.
Copy the new vote for each candidate from the vote record forms onto the result sheet, and the new non-transferable vote from the exclusion forms onto the result sheet.
Add up the new total vote on the result sheet and check that this agrees with the original total valid vote.
Completion of the count
If a proposed exclusion of one or more candidates would leave only the same number of continuing candidates as there are places remaining unfilled, all such continuing candidates shall be deemed to be elected.
If, at any point in the count, the number of candidates deemed to be elected is equal to the number of places to be filled, no further transfers of papers are made, and the remaining continuing candidate(s) are formally excluded.
The count is now completed.
Declare elected all those candidates previously deemed to be elected.
Notes
Calculation of the total active vote may be simplified if the Count Control Form (beige) is used. This form enables the Returning Officer to keep a continuous check on the number of votes which are required for election of a candidate at any point in the count, by deducting the quotas (or actual votes if less) of the candidates deemed elected, and the total of non-transferable votes, from the total valid vote, to give the total active vote.
If, when candidates should be deemed elected under sections 5.1.7, 5.3.13 or 5.4.9, two or more have the same number of votes, and there are not sufficient places left for them all, then the one or more to be deemed elected shall be selected in descending order of votes at the first stage or at the earliest point in the count, after the transfer of a batch of papers, where they had unequal votes. If, however, their votes have been equal at all such points, then none of them shall be deemed elected at that stage.
If a re-count is conducted where a decision has been determined by lot, and the relevant votes are still equal in the recount, the earlier determination shall still hold.
These rules refer to the various forms published by the Electoral Reform Society. The use of these forms is optional, but where they are used, the various options should be made easier.
Glossary of terms in alphabetical order
Batch: a bundle containing all the papers of one value in a transfer.
Candidate's vote: the value of voting papers credited to a candidate at any point in the count.
Continuing candidate: a candidate not yet deemed elected or excluded.
Count Control form (beige): a form designed to be used to keep a continuous note of the total active vote, and hence the vote required for election of a candidate at any point in the count.
Counting slip (green): a slip inserted with a bundle of voting papers, showing the stage at which the papers are transferred, the name of the candidate to whom they are transferred, the number of papers in the bundle, and the transfer value of each paper.
Deemed elected: status of a candidate who is elected subject to formal confirmation.
Exclusion form (blue): a form showing the distribution of batches of papers in descending order of transfer value from one or more excluded candidates to continuing candidates.
First preference: this is shown by the figure "1" standing alone against only one candidate on a voting paper; or the name or code of a candidate entered on a voting paper as first preference.
Invalid paper: a voting paper on which no first or only preference is expressed, or on which any first preference is void for uncertainty.
Next available preference: the next subsequent preference in order, passing over earlier preferences for candidates already deemed elected or excluded. There is no next available preference where the next sequential preference for a continuing candidate is uncertain.
Non-transferable difference: the difference between the value of a surplus and the total new value of the papers transferred, which arises from ignoring the remainder when calculating the transfer values to two decimal places.
Non-transferable paper: a voting paper on which no next available preference for a continuing candidate is expressed, or on which any next available preference is void for uncertainty.
Non-transferable vote: the value credited as non-transferable at any point in the count.
Quota: the vote which, if attained by as many candidates as there are places to be filled, leaves at most a quota for all other candidates; the total valid vote divided by one more than the number of places to be filled, or a lesser value calculated as in paragraph 5.4.8.
Result sheet (white): a sheet showing the vote credited to each and every candidate, and the non-transferable vote at successive stages of the count.
Stage of the count: the determination of the first preference vote for each candidate (first stage)
or the transfer of a surplus
or the exclusion of a candidate, or two or more candidates at the same time, and the transfer of their votes.
Subsequent preferences: shown by the figures "2", "3", etc., standing alone against different candidates on a voting paper; or the names or codes of candidates entered in order on a voting paper as second, third, etc., preferences.
Surplus: the amount by which a candidate's vote exceeds the quota.
Surplus form (pink): A form showing the calculation of the transfer value and the distribution of transferable papers from a candidate deemed elected to continuing candidates.
Total active vote: the sum of the votes credited to all continuing candidates, plus any votes awaiting transfer.
Total valid vote: the total number of valid voting papers.
Transfer value: the value, being unity or less, at which a voting paper is transferred from an elected or an excluded candidate to a continuing candidate. Where counting slips are not used, it is recommended that this value be marked on each paper at the time of transfer.
Transferable paper: a voting paper which, having been allocated to a candidate, bears a next available preference for a continuing candidate.
Valid voting paper: a voting paper on which a first or an only preference is unambiguously expressed.
Value: the value of a voting paper is unity, or a lower value at which it was last transferred.
Vote record form (yellow): a form showing the vote credited to any one candidate, or showing the non-transferable vote, at successive stages of the count.


