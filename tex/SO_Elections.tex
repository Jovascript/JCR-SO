\chapter{Elections}
\hspace*{-10pt}\textit{NB: This is an Agreed Standing Order}

\npara The Executive Committee shall take all reasonable steps to ensure that Elections of Officers are free, fair and properly conducted, including allowing the Returning Officer sufficient autonomy to fulfil their responsibilities independently.
\npara No Candidate shall take part in any part of the administration of the Election procedure.
\npara After an Election to the Executive Committee (including a By-Election), the Returning Officer shall produce a report on the conduct of the Election (or Elections) and submit it to the College (via the relevant College Officer), the Executive Committee and the General Meeting within twenty Term-Time Days of the Election (or Elections).  Provided they remain an Ordinary Member, they shall present it to the next OGM after submission.
\npara At the end of each Full Term, the Returning Officer shall produce a report on the conduct of the OGM Elections in that Term and submit it to the College (via the relevant College Officer), the Executive Committee and the General Meeting.  Provided they remain an Ordinary Member, they shall present it to the first OGM of the subsequent Term.
\section{Returning Officer}
\npara The Returning Officer shall be a Non-Committee Officer.  For the avoidance of doubt, the office may be held in conjunction with any other office.
\npara The Returning Officer is responsible for overseeing the free, fair and proper conduct of all Elections and procedures relating to them, under the supervision of the Executive Committee.
\npara In the first instance, the Returning Officer shall be responsible for interpreting the Constitution and Standing Orders, insofar as they relate to Elections and Referenda.  The Returning Officer shall ensure that any dispute in interpretation is referred to the Executive Committee, in accordance with Article \ref{Art:Name}.
\npara In the absence of the Returning Officer, the office shall be taken over by the President, the Vice President or the Treasurer in that order.
\npara The Returning Officer shall attend all Ordinary General Meetings and shall update the General Meeting on their work and answer any relevant questions.
\npara The Returning Officer shall never be a Candidate.
\npara The Returning Officer shall be elected in an OGM Election in the third OGM of each Term, to serve for the duration of the following Term.
\npara In the event that the elected Returning Officer leaves office before their term is due to end, an OGM Election shall be held at the next OGM to fill the position.
\npara In the event that a Returning Officer is not elected at an OGM Election for the office (i.e. RON is deemed to have won), an OGM Election shall be held at the next OGM to fill the vacancy.
\npara In addition to these duties, the Returning Officer shall have such further duties as may be required by the Standing Orders or reasonably required by the Executive Committee, provided that they do not actually or potentially conflict with the duties set out in this Standing Order.
\section{Eligibility for Candidature}
\npara Subject to S.O. 4.5.6. and the following clauses, any Ordinary Member shall be eligible to stand for Election to any post on the JCR Committee, or where applicable, any seat on Sub-Committees or Standing Committees.
% * <julesdesai14@gmail.com> 2018-09-25T14:42:40.627Z:
% 
% > S.O. 4.5.6
% Need to link
% 
% ^.
\npara Each position is to be held by one person only.
\npara No Candidate is permitted to stand for more than one Executive Committee post in the same Election.  The Presidential Election and Executive Committee Elections shall be deemed to be separate Elections in the application of this Standing Order.
\npara No member of the Executive Committee may stand for re-election or election to another Executive Committee post except with the approval of their candidacy in advance by Special Resolution of the General Meeting.
\npara All Candidates must adhere to the rules pertaining to the position for which they are standing concerning Manifestos and Hustings.
\section{Schedule of Nominations, Hustings and Elections}
\npara Nominations for an office elected by OGM Election shall close immediately before Hustings for the position are held in the Ordinary General Meeting.
\npara There shall be Scheduled Elections for every Non-Executive Officer of the JCR Committee. 
\npara The Scheduled Election shall not take place if a By-Election is held in either of the two OGMs prior to the meeting in which the Scheduled Election is due to occur.
\npara In each Academic Year, for the Annual Election of the President, Hustings shall take place in the third week of Trinity Term and the Polling shall take place on Friday of the third week of Trinity Term.
\npara In each Academic Year,  for the Annual Elections of the other Executive Committee Positions, Hustings shall take place in the fourth week of Trinity Term and the Polling shall take place on Friday of the fourth week of Trinity Term. 
\npara The Returning Officer shall give notice of the opening of nominations for Presidential and Executive Committee Elections no fewer than five days before they open.
\section{Manifestos}
\npara Manifestos shall be required only for Elections of the members of the Executive Committee.
\npara The Returning Officer shall solicit Manifestos no fewer than two weeks prior to the closure of nominations.
\npara Manifestos must have been received by the Returning Officer prior to the closure of nominations and be accompanied by a relevant tutor's written or emailed permission for the Candidate to stand.
\npara Each Manifesto must be in A4 format and must contain a photograph of the Candidate. The Returning Officer must receive three paper copies of each Manifesto as well as a digital copy.
\npara One paper Manifesto shall be placed in the College archive; a digital copy shall be placed on the JCR Website for the future reference of the JCR.
\npara The Returning Officer shall prominently display in College two copies of the Manifestos of all Candidates within 24 hours of the closure of nominations. Copies of the Manifesto should also be posted on the JCR Website within 24 hours of the closure of nominations.
\npara Manifestos must not contain any defamatory material.
\npara Manifestos must not refer to or show symbols of any political organisation, unless a Candidate is demonstrating their suitability for the role in question due to skills acquired through involvement with such organisations.
\npara The Returning Officer shall ensure that Manifestos remain displayed until the close of Polling.
\section{Hustings}
\npara Hustings shall be a forum for the questioning and examination of Candidates on issues pertaining to the position for which they are standing for the purpose of judging fairly the merits of all Candidates.
\npara All Candidates must attend and take part in Hustings except under exceptional circumstances at the discretion of the Returning Officer.
\npara Hustings for the Elections of members of the Executive Committee shall be held between the closing of nominations and the beginning of Polling. The Returning Officer shall give notice to Ordinary Members of the time and location of Hustings for the Elections of members of the Executive Committee no less than 72 hours before they are due to take place.
\npara Hustings for positions on a Standing Committee requiring Election by the Founding Resolution thereof shall be held during the designated General Meeting.
\npara The Returning Officer shall act as Chair of Hustings and maintain an atmosphere of order and fairness.
\npara All questions must be put through the Chair of Hustings at whose discretion any question may be disallowed on the grounds of irrelevance, malice or partiality.
\npara All questions must be addressed to all Candidates.
\npara Immediately prior to Elections to Non-Executive JCR Committee offices, the incumbent will be expected to give a brief summary of their experience in the role.  In the event of the incumbent being absent, a written submission should be sent to the Secretary to the General Meeting before the meeting in which the Election is to be held for the office in question. The Candidates will then be asked to come forward and introduce themselves, prior to Hustings.
\section{Canvassing}
\npara All Candidates are entitled to canvass and produce Election publicity material in accordance with the Constitution and Standing Orders.
\npara Election publicity material must not contain any defamatory material.
\npara Election publicity material must not reference or show symbols of any political organisation, unless a Candidate is demonstrating their suitability for the role in question due to skills acquired through involvement with such organisations.
\npara All Election publicity material is subject to the scrutiny of the Returning Officer. If necessary, the Returning Officer shall remove any such material found to be in contravention of S.O. 4.10.
% * <julesdesai14@gmail.com> 2018-09-25T14:48:56.238Z:
% 
% > S.O. 4.10
% link
% 
% ^.
\npara No Election publicity material is allowed within sight of the Ballot Box on the day of Polling with the exception of the Manifestos posted by the Returning Officer.
\npara Candidates must not spend money on canvassing (nor receive canvassing support of monetary value), except for any cost incurred by the printing of canvassing materials required by the Constitution and Standing Orders, or with the express authority of the Returning Officer.
\subsection{Restrictions on Members}
\npara No Member shall take any action which might unfairly prejudice the outcome of the Election and the Returning Officer shall issue Guidance accordingly.  The general principle is that no Member should communicate their opinion by a means which could not reasonably be used by any Member who wished to do so.
\npara If necessary for the free, fair and proper conduct of the Election, the Returning Officer may require any Member to withdraw a statement that they have made or to apologise for an action that they have taken. 
\section{Polling}
\subsection{General}
\npara Only Ordinary Members may vote in Elections.
\npara Postal Votes shall not be accepted, except that Ordinary Members who are on a year abroad, or who are suspended from their studies, may vote in Elections to the Executive Committee using a secure, secret online voting system chosen by the Returning Officer.
\npara All Elections shall include the option to re-open nominations (RON).
\npara No Member shall be obliged to use any or all of the votes at their disposal.
\npara A Candidate may withdraw from the Election at any time up until the beginning of Polling.
\subsection{Executive Committee Elections}
\npara For Elections to the Executive Committee, the Returning Officer shall publicise the date and venue of Polling no fewer than five days before it is to take place. The Returning Officer shall post a guide to the voting procedure in a suitable place for the duration of Polling. The Returning Officer shall also notify via email members on a year abroad, or who are suspended from their studies, of the date of Polling, and of the online voting system to be used, no fewer than five days before Polling is to take place. The Returning Officer shall send such members a further email up to 24 hours before the start of Polling, providing the web address of the relevant online voting system and a guide to the voting procedure.
\npara Elections to the Presidency or Executive Committee shall be by secret ballot. Polling shall be open from 8.30am to 8.30pm.
\npara The Returning Officer shall enlist scrutineers to supervise the Election on the day of Polling. Scrutineers may explain the completion of the Ballot Paper but shall not, whilst supervising Polling, advise or instruct any Member concerning the casting of their vote.
\npara Elections to the Executive Committee shall be either by Individual Election (for election to a single office) or by Joint Election (for election to two or more connected offices).
\npara The procedure for the election of members of the Executive Committee by Individual Election shall be by Alternative Vote as set out below:
\begin{enumerate}
\item Voting shall be by the system of a single alternative vote and must be in order of preference, expressed numerically with 1 expressing the first preference, 2 the second preference and so on for as many preferences as the voter wishes to express or until they have no further preferences.
\item The ballot shall list all Candidates in alphabetical order by surname, and RON which always appears last.
\item Any indication of a vote for a Candidate, where there is only one such indication, shall be deemed a first preference vote for that Candidate.
\item Any deliberate mark made on a ballot paper outside a box invalidates that ballot paper, which shall be considered spoilt.
\item Voters shall be entitled to exchange an accidentally spoilt ballot paper for a new ballot paper with the scrutineers, before putting any ballot paper in the ballot box.
\item Each Candidate's first preference votes shall be counted, discarding any ballot papers that are spoilt, blank or void. The option to re-open nominations (RON) shall be treated in the same way as a Candidate.
\item If no Candidate has more than 50\% of the valid votes cast, the Candidate with the fewest first preference votes shall be excluded and the ballot papers redistributed according to the subsequent preference.
\item Ballot papers redistributed under vii shall continue to be counted at their full value. Ballot papers which do not express a further preference shall be excluded.
\item Stages vi, vii and viii shall be repeated until one Candidate (or RON) has over 50\% of the remaining votes or there is only one Candidate remaining. This Candidate shall be declared elected.
\item Should RON gain over 50\% of the valid votes at any stage of the counting procedure then the procedure detailed in S.O. 4.11.2.10. shall be followed.
\item In the event of a tie, the Candidate with the highest number of first preferences shall be declared elected. If a tie is still the result, then the Candidate with the greatest number of second preferences shall be declared elected, and so on. If the result is still a tie once all the preferences have been taken into account, RON shall be deemed to have won and a second Election shall be held following the procedure detailed in S.O. 4.11.2.10.
\end{enumerate}
\npara The procedure for the election of members of the Executive Committee by Joint Election shall be by Single Transferable Vote as set out below:
\begin{enumerate}
\item Voting shall be by the system of a Single Transferable vote and must be in order of preference, expressed numerically with 1 expressing the first preference, 2 the second preference and so on for as many preferences as the voter wishes to express or until there are no other Candidates.
\item The ballot shall list all Candidates in alphabetical order by surname, and RON which always appears last.
\item Any indication of a vote for a Candidate, where there is only one such indication, shall be deemed a first preference vote for that Candidate.
\item Any deliberate mark made on a ballot paper outside a box invalidates that paper, which shall be considered spoilt.
Voters shall be entitled to exchange an accidentally spoilt ballot paper for a new ballot paper with the scrutineers, prior to putting any ballot paper in the ballot box.
\item The count shall be conducted in accordance with the ERS97 system explained in Appendix F.
\item If RON is amongst the Candidates deemed elected, the result of the Election shall be annulled and the procedure in S.O. 4.11.2.10. shall be followed.
\end{enumerate}
\npara Votes shall be counted by the Returning Officer, who shall enlist at least two members of the Executive Committee to assist in the counting.
\npara All Candidates and any member of the JCR Committee shall be entitled to observe the counting of the votes, provided that they do not interfere with the counting.
\npara Results of the Presidential or Executive Committee Elections shall be announced in Front Quad within 3 hours of the close of Polling, except in extraordinary circumstances.
\npara If RON wins in the case of the Presidency or an Executive Committee position, nominations shall re-open for a period of three Term-time Days, Hustings shall occur on the fourth Term-time Day and Polling shall be held seven Term-time Days after the original Election. All rules concerning the content and display of Manifestos and the conduct of Hustings and Polling shall be adhered to.
\npara If no Candidate stands by close of nominations, RON shall be deemed to have won.
\subsection{OGM Elections}
\npara Non-Executive Officers of the JCR Committee shall be elected by OGM Election.
\npara Elections shall be by show of hands, in the absence of the Candidates, at an Ordinary General Meeting of the JCR after the Hustings. Counting shall be by the Returning Officer, the Chair of the General Meeting and the Secretary to the General Meeting. If requested by at least two Ordinary Members, present and voting, the Election shall take place by secret ballot.
\npara The Returning Officer shall ensure that Ordinary Members are given suitable notice of the Elections.
\npara The Candidate with the most votes shall be deemed elected.
\npara A RON victory or a tie in the Election shall result in the re-opening of Nominations. A further Election shall be held at the next OGM.
\npara If no Candidate stands by close of nominations, RON shall be deemed to have won.
\section{By-Elections}
\subsection{JCR Committee}
\npara Should a Non-Executive JCR Committee office become vacant an OGM Election shall be held at the following Ordinary General Meeting to fill the vacancy.
\subsection{Executive Committee}
\npara Should an Executive Committee office become vacant, the Returning Officer shall supervise the holding of a By-Election within seven Term-time Days.
\npara Nominations shall be opened immediately upon the Returning Officer being informed that the position is vacant. The Returning Officer shall give notice to all Ordinary Members of the vacancy within three hours of being informed. Nominations shall remain open for three Term-time Days.
\npara All Candidates must meet the eligibility criteria set out in S.O. 4.6. and provide Manifestos adhering to the rules of content and display set out in S.O. 4.8. Canvassing in accordance with S.O 4.10. is allowed.
\section{Terms of Office}
\npara Members of the Executive Committee shall be elected to serve for the subsequent Academic Year.  In the event of a By-Election for an Executive Committee office after the Academic Year has begun, the person elected shall serve for the remainder of the Academic Year.
\npara Non-Executive Officers of the JCR Committee shall be elected to serve serve until the next Scheduled Election.
\section{Complaints}
\npara Complaints concerning Elections shall be dealt with in accordance with S.O. 10.
\npara In addition to any other remedies, the following remedies shall be available to provide effective remedy for a Complaint concerning an Election.
\begin{enumerate}
\item Require a Candidate to apologise.
\item Require a Member to withdraw or remove a statement or other publicity.
\item Annul the result of an Election and require a By-Election to take place.
\item For an Executive Committee Election, if the Polling has not yet completed, disqualify a Candidate from the Election.
\end{enumerate}
\npara Any remedy implemented must be proportionate for the purpose of ensuring that Elections are free, fair and properly conducted.




