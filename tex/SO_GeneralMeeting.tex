\chapter{The General Meeting}
\hspace*{-10pt}\textit{NB:This is an Internal Standing Order.}
\section{General}
\npara Only Ordinary Members have the right to attend, speak and Vote at General Meetings. Honorary Members may attend and speak at General Meetings, but may not Vote.  At the discretion of the Chair, strangers may be permitted to observe. Strangers may never Vote.
\npara The positions of Chair of and Secretary to the General Meeting shall never be held by the same individual.
\npara All business of the General Meeting shall be subject to all Procedural Motions.
\npara Any Ordinary Member may raise a point of order with the Chair at any time, provided the point of order is raised at an appropriate time.  It is not appropriate to interrupt someone speaking with a point of order unless the point of order concerns what is being said.
\npara Unless the Constitution and Standing Orders require otherwise, the General Meeting shall make decisions and resolutions by Ordinary Resolution on a motion at one meeting.
\npara The Quorum of the General Meeting is 30 Ordinary Members. The Chair is responsible for ensuring that a Quorum is present.
\section{Convention of General Meetings}
\npara Ordinary General Meetings shall be held four times each term, on the Sundays of 1st, 3rd, 5th and 7th Week and shall be convened by the Chair of the General Meeting.
\npara In accordance with S.O.6.2.3 and Article \ref{Art:GeneralMeeting}, the Chair of the General Meeting shall convene an Extraordinary General Meeting if:
\begin{enumerate}
\item they are requested to do so by either the Executive Committee or the JCR Committee;
\item they are requested to do so by the General Meeting; or
\item they are requested to do so by petition of at least thirty Ordinary Members.
\end{enumerate}
\npara A request to convene an Extraordinary General Meeting must include the proposed Motion which is to be considered by the meeting.  The Chair may refuse to convene a General Meeting when requested to do so if the proposed Motion:
\begin{enumerate}
\item contains defamatory material;
\item contains substantively the same material as a Referendum in the same Academic Year;
\item relates to a matter that could have been dealt with at an OGM because it was reasonably foreseeable;
\item relates to a matter that could be dealt with at a future OGM because it is not urgent; or
\item contains substantively the same material as a Motion proposed previously in the Term (except motions requiring the assent of two consecutive Ordinary General Meetings) or at the same meeting.
The Chair must refuse a request to convene an Extraordinary General Meeting if the proposed Motion seeks to overturn the result of a Referendum held less than three years ago other than by Referendum.
\end{enumerate}
\npara If an Extraordinary General Meeting is requested, the Chair must ensure that Ordinary Members are given at least 24 hours notice of the meeting.  The Chair shall convene the meeting within three Term-time Days of receiving the request unless the Constitution and Standing Orders require otherwise.
\section{Order of business: Ordinary General Meeting}
\npara In Ordinary General Meetings, Business shall be transacted in the following order.  In unusual circumstances, the Chair may exercise their discretion to vary this order.
\begin{enumerate}
\item Minutes of the previous Ordinary General Meeting.
\item Matters arising from the Minutes of the previous Ordinary General Meeting.
\item Ratification of the Minutes of the previous Ordinary General Meeting.
\item President's Business.
\item Committee Business.
\item Ratifications of any Dismissals or Minutes of Emergency or Extraordinary General Meetings held since the previous Ordinary General Meeting.
\item Questions to the JCR Committee.
\item Any Other Business.
\item Hustings and Elections.
\item Motions.
\end{enumerate}
\section{Order of business: Extraodinary General Meeting}
\npara In Extraordinary General Meetings, Business shall be transacted in the following order.  In unusual circumstances, the Chair may exercise their discretion to vary this order.
\begin{enumerate}
\item First Reading of the motion.
\item Debate on the motion.
\item Final Reading of the motion.
\item Vote.
\end{enumerate}
\section{Agenda: Ordinary General Meeting}
\npara The Secretary to the General Meeting shall give notice of Ordinary General Meetings to all Ordinary Members no fewer than five days before they are to be held, in accordance with Article \ref{Art:GeneralMeeting}.  The Chair of the General Meeting shall provide the Secretary with the information that they need in good time in order to do this.
\npara Motions must reach the Secretary in writing by 6pm four days before the meeting.
\npara For the first OGM of each Term, the Secretary shall add to the Agenda, under the business of the relevant Officer, the previous Term's Committee report (prepared in accordance with S.O. 2.3.2.1) and the Accounts of the previous Term and the Budget for the coming Term (prepared in accordance with S.O. 8.3.3).
\npara The Secretary shall decide the order of the motions and give notice of the Agenda for the meeting to the Ordinary Members by 12pm three days before the meeting.
\npara All Ordinary Members shall receive a digital copy of the Agenda.
\npara Every member of the JCR Committee shall have the right to speak in the section of the Ordinary General Meeting designated as ``Committee Business'' in S.O. 6.3.1.v. and to submit any business they may have in advance.
\npara Emergency motions must reach the Secretary in writing at least three hours before the start of the meeting and may be added to the Agenda of the meeting at the discretion of the Secretary. Emergency motions should deal only with urgent or unforeseen business having occurred since the closure of the deadline for standard motions.
\npara In exceptional circumstances, the Chair of the General Meeting may, at their own discretion, accept Emergency motions at any point up to and during the Meeting.
\npara The Chair may change the order of motions on the Agenda.
\npara The Secretary may refuse a motion on the grounds that it contains defamatory material or contains substantively the same material as a Referendum in the same Academic Year or as a motion proposed previously in the term (except motions requiring the assent of two consecutive Ordinary General Meetings) or at the same meeting. The Secretary must refuse a Motion if it seeks to overturn the result of a Referendum held less than two years ago, other than by Referendum.  The Secretary must report to the Ordinary General Meeting the reasons for refusing a motion if challenged on the issue in "Questions to the JCR Committee" (S.O. 6.3.1.vii). At the discretion of the Chair a Vote may be held to decide whether the motion shall be placed on the Agenda; resolution is by simple majority vote and the Chair shall have the casting vote.
\section{Agenda: Emergency and Extraordinary General Meetings}
\npara Each Extraordinary General Meeting may consider only one Motion.
\npara The Secretary to the General Meeting shall give notice of the motion proposed at an Extraordinary General Meeting to the Ordinary Members at least 24 hours in advance of the Meeting.
\section{The Chair of the General Meeting}
\npara The President shall be Chair of the General Meeting. In their absence, the Chair shall be taken by the Treasurer, Access and Equal Opportunities Representative, Academic Affairs Representative, either Welfare Officer or either Entertainment Representative in that order.
\npara The Chair shall be responsible for the smooth running of the General Meeting, maintaining order, preserving an atmosphere of consideration, reading motions and ensuring that the Constitution and Standing Orders are respected.
\npara Anyone wishing to speak must indicate their desire by raising their hand. The Chair shall decide the order of priority in speaking.
\npara The Chair may disallow questions on the grounds of irrelevance or unfairness.
\npara The Chair may interrupt debate on the floor in order to play an inquisitorial role in the Debate.
\npara The Chair shall have the right to caution any person at the General Meeting for disruptive conduct.
\npara During a General Meeting, the Chair shall be the sole interpreter of the Constitution and Standing Orders.  The Chair must interpret within any interpretation made by the Executive Committee.  The Executive Committee shall have the power to overturn an interpretation of the Chair.
\npara Should the Chair be placed in a position of conflict of interest in Debate, they must relinquish the Chair for the duration of that Debate (until opposition is withdrawn or a Vote is taken). The Chair shall then be taken by the Treasurer, Access and Equal Opportunities Representative, Academic Affairs Representative, either Welfare Officer or either Entertainment Representative in that order. The original Chair may return afterwards.
\section{The Secretary to the General Meeting}
\npara The Vice President shall act as Secretary to the General Meeting. In their absence, the Academic Affairs Representative, Access and Equal Opportunities Representative, Treasurer, either Welfare Officer or either Entertainment Representative shall take over secretarial duties, in that order.
\npara The Secretary shall be responsible for compiling the Agenda, reading the minutes, keeping a true and fair record of the business of the General Meeting (in accordance with Article \ref{Art:Minutes}), compiling amendments, and counting and recording Votes.
\npara Should the Secretary be placed in a position of conflict of interest in debate they must relinquish secretarial duties for the duration of that debate (until opposition is withdrawn or a Vote is taken). They shall be taken over by the Academic Affairs Representative, Access and Equal Opportunities Representative, Treasurer, either Welfare Officer or either Entertainment Representative in that order. The original Secretary may return afterwards.
\npara The Secretary shall be responsible for publishing the minutes within 48 hours of the close of the relevant meeting.
\section{Motions: All General Meetings}
\subsection{General}
\npara All motions require a named Proposer and named Seconder who are Ordinary Members, and are present at the meeting. In the absence of the named Proposer or Seconder, a replacement may be appointed by the Chair.
\npara Unless otherwise prescribed by the Constitution or Standing Orders, a simple majority vote of those present at the General Meeting shall be sufficient to pass all motions.
\npara Hereinafter, the Proposer, Seconder to the Proposition, Opposer and Seconder to the Opposition shall be referred to as ``the Speakers''.
\npara The section of the General Meeting entitled ``Motions'' as set out in S.O. 6.3.1.xi. shall follow the procedures for First Reading, Debate, Amendments and Voting described below.
\npara Parts of a motion are enforceable only if they are a resolution of the motion.
\npara Unless otherwise provided for by the Standing Orders or Constitution, all actionable resolutions of a motion must either mandate a well-defined group of individuals to performed well-defined actions, hereby referred to as Mandating Resolutions, amend the Standing Orders or Constitution, or direct the allocation of JCR funds.
\npara All Mandating Resolutions must implicitly or explicitly identify a Responsible Officer, who must update the JCR on the progress of the motion at each OGM. For the avoidance of doubt, resolutions mandating a single Ordinary Member to perform an action implicitly nominate this individual as the Responsible Officer.
\npara Responsible Officers may delegate their future responsibility to another consenting Ordinary Member. Should the Responsible Officer no longer be an Ordinary Member or unable to fulfil their duties, their duties shall be delegated to another Ordinary Member at the discretion of the President.
\npara All Mandating Resolutions of the JCR must implicitly or explicitly include conditions for fulfilment.
\npara A Mandating Resolution can specify a time limit, no longer than 1 year. If not specified, the default time limit is 1 year.
\npara If a Mandating Resolution is not fulfilled before the time limit, the resolution loses effect, and the Responsible Officer must report on its failure at the next OGM.
\npara The Responsible Officer shall, in conjunction with the Information Technology Officer, ensure the up-to-date status of the resolution for which they are responsible is made available to all members.
\npara All resolutions which amend the Standing Orders or Constitution must contain the explicit directions detailing the location and textual content of all changes.

\subsection{Procedure of the First Reading}
\npara There shall be a First Reading of each of the motions before the General Meeting to establish if there is any opposition to them. For this purpose an Amendment counts as opposition. Short factual questions may be accepted at the discretion of the Chair.
\npara If there is no opposition to a motion and no member present desires further discussion, it shall be passed immediately \textit{nemine contradicente}.
\npara In the event of opposition to a motion, an Opposer and a Seconder to the Opposition shall be appointed from the floor by the Chair.  The Opposer and Seconder to the Opposition may be the same Ordinary Member.
\npara In the event that there is no opposition to the motion but further discussion is desired by a Member, the Chair may take initial points and questions from the floor in accordance with S.O. 6.9.3.2, S.O. 6.9.3.3 and S.O. 6.9.3.4.
\npara This preliminary discussion shall end if there is opposition to the motion in accordance with S.O.8.9.2.3. and Debate shall begin.  
\npara The Chair may end the preliminary discussion by a call for opposition if they are satisfied that no new or useful information remains to be offered or if more than seven minutes have elapsed. If there is no opposition at this point, the motion shall be passed immediately \textit{nemine contradicente}.
\subsection{Procedure of the Debate}
\npara The Proposer of the motion or Amendment shall have the right to begin the debate with a speech, followed by the Opposer, the Seconder to the Proposition and the Seconder to the Opposition. The Chair shall have the right to curtail speeches longer than five minutes in duration.
\npara The debate shall then pass to the floor where, when recognised by the Chair, members may make observations, propose questions to one or more of the Speakers or propose an Amendment under S.O. 6.9.4.
\npara All input from the floor is at the discretion of the Chair who shall decide the order of speaking and whether any questions may be disallowed on the grounds of irrelevance or unfairness.
\npara Any member speaking may, at their discretion, accept points of information.
\npara Once the Chair is satisfied that no new or useful information remains to be offered to debate, the Chair shall invite the Opposer and then the Proposer to present a speech of summation before the Vote is taken.
\subsection{Amendments}
\npara An Amendment may be proposed at any point either during the first reading of the motion or its debate up until the point of speeches of summation.
\npara Debate upon the motion is suspended immediately an Amendment to the motion is proposed.
\npara Amendments require both a Proposer and a Seconder, both of whom are Ordinary Members, unless there is no opposition to the Amendment from Ordinary Member in attendance.
\npara Amendments may not change the issue with which the motion is concerned.
\npara The specific wording of the Amendment must be immediately decided upon and will be read by the Secretary to the General Meeting.
\npara The Proposer and Seconder of the motion to be amended may accept the Amendment, provided that there is no opposition from the floor; in this eventuality the normal course of debate is resumed immediately.
\npara If the Amendment is not accepted, the Chair shall recognise an Opposer and Seconder to the Opposition, who may be the Proposer and Seconder of the motion and the Amendment will be debated according to S.O. 6.9.3 and Voted upon according to the procedure laid down in S.O. 6.9.5.  As in S.O. 6.9.2.3. the Opposer and Seconder to the Opposition may be the same Ordinary Member.
\npara Further Amendments to the motion cannot be proposed until a Vote has taken place on the current Amendment.
\npara The wording of the Amendment may be changed, at any point prior to Voting on the Amendment, at the discretion of the Proposer of the Amendment.
\npara All Amendments supported by a simple majority vote of those present shall become part of the motion.
\npara The Proposer and Seconder of the motion shall have the right to withdraw their support from the motion as amended. The Chair may appeal to the floor for another Proposer and Seconder who then assume control of the amended motion.  If either place cannot be filled, then the motion fails.
\subsection{Voting}
\npara First, the Chair or the Secretary shall read in full the motion or Amendment or explain fully the issue on which the Vote is to be held. A Vote shall be said to have begun when this process is complete.
\npara Voting shall by default be by show of hands.
\npara Counting shall be by at least two of the Chair, the Secretary to the General Meeting and the Returning Officer.
\npara No proxy voting shall be permitted.
\npara No Postal Votes of any kind shall be permitted.
\npara In accordance with the Constitution, in the event of a tied Vote, the Chair shall have a second or casting vote.
\section{Procedural Motions}
\npara Unless stated to the contrary below, the following Procedural Motions may be put at any General Meeting at any point within the meeting by any member:
\begin{enumerate}
\item \textit{Quorum Count}: the Quorum Count shall proceed automatically if allowed by the Chair; the Quorum of the General Meeting is thirty members; the General Meeting shall be assumed to be quorate unless a Quorum Count is requested; if found to be inquorate, the General Meeting shall adjourn for five minutes, or for up to half an hour if the Chair so determines; if quoracy is still lacking after reconvening then this shall be recorded in the Minutes and the meeting shall proceed in an advisory capacity only; further motions passed shall not become policy until the Minutes of the inquorate meeting are passed at the next Ordinary General Meeting.
\item \textit{Challenge to the Ruling of the Chair}: the Proposer of this motion must state what they think that the ruling should be; resolution is by simple majority vote of those present.
\item \textit{Secret Ballot}: requiring the support of two Ordinary Members present and voting to pass, this Procedural Motion shall force any Vote in the process of being held, but having not yet begun, to be held by Secret Ballot. The conduct of the Secret Ballot shall be fair under the charge of the Secretary to the General Meeting and the Returning Officer, and Voting must be done either through use of ballot papers, or a method unanimously supported by all Ordinary Members present. If possible the General Meeting should continue whilst the result of the Vote is determined. Secret Ballot shall overrule Vote by Physical Division.
\item \textit{No Confidence in the Chair}: this requires the support of two-thirds of those present to be passed; if passed, the member acting as Chair shall vacate the Chair for the remainder of the meeting; the Chair shall pass to the next suitable member according to SO6.7.1.
\item \textit{Expulsion of Strangers}: pursuant to S.O. 6.1.1 and requiring a simple majority vote to pass, Expulsion of Strangers shall result in all non-Members being asked to leave the General Meeting.
\item \textit{Expulsion of a Member}: any member may be expelled from the meeting for disruptive conduct with the support of two-thirds of those present.
\item \textit{Removal of Silent Observation}: any member may propose this Procedural Motion in favour of any stranger present at the General Meeting; requiring the support of a majority of those present. If passed, the stranger in question shall be allowed to contribute to the debate but not to Vote.
\item \textit{Adjournment}: A member may move that the meeting be adjourned until a specific time; adjournment requires the support of two-thirds of those present.
\item \textit{Deferral}: by simple majority vote, a motion on the agenda of an Ordinary General Meeting may be deferred to the subsequent Ordinary General Meeting. Motions before Emergency or Extraordinary General Meetings may not be deferred.
\item \textit{Refer to Referendum}: by simple majority vote, resolution of a motion on the agenda of the meeting may be determined by a Referendum rather than a Vote in the meeting, in accordance with Article \ref{Art:GeneralMeeting} and S.O. 5.
\item \textit{Voting in Parts}: requiring a simple majority vote to pass, this Procedural Motion shall force the meeting to debate and vote upon the motion under consideration in the parts in which it was submitted; Voting in Parts does not force the meeting to move to a vote.
\item \textit{Vote by Physical Division}: requiring a simple majority vote to pass, this Procedural Motion shall force any Vote in the process of being held, but having not yet begun, to be held by Physical Division.  This may be overruled by Secret Ballot.
\item \textit{Move to a Vote}: requiring a simple majority vote to pass, Move to a Vote will force debate to be curtailed and the General Meeting to proceed directly to speeches of summation and then to a Vote.
\end{enumerate}
\npara Procedural Motions shall be taken in the order of precedence as they appear above.
\npara The Chair shall be obliged to hear any Procedural Motion put and the Proposer's reasons for proposition.
\npara The Chair shall have the right to disallow all Procedural Motions except S.O. 6.10.1.iv ``No Confidence in the Chair'' and S.O. 6.10.1.ii ``Challenge to the Ruling of the Chair''. The Chair may only disallow a ``Quorum Count'' S.O. 6.10.1.i if such a count has been allowed to take place in the previous ten minutes and the Chair is reasonably satisfied that a Quorum is present.
\npara All Debate shall be suspended immediately a Procedural Motion is proposed.
\npara When a Procedural Motion is put and allowed by the Chair, the Chair shall repeat the Procedural Motion which shall then take precedence over all other business except a Procedural Motion higher in the order of precedence. After the Proposer has stated their reasons, the Chair may appoint an Opposer to the motion at their discretion except in the case of Move to a Vote and Quorum Count where there shall be no Debate. All Procedural Motions except the Quorum Count and Move to a Vote may be debated in line with S.O. 6.9.3. if there is opposition, at the discretion of the Chair.
% * <julesdesai14@gmail.com> 2018-09-25T19:55:24.076Z:
% 
% these were numbered 6.10.1.1 - 6.10.1.6 in the google doc. Was this intentional?
% 
% ^.




