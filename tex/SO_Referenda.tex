\chapter{Referenda}
\textit{NB: This is an Agreed Standing Order}
\npara The Returning Officer shall be responsible for overseeing all Referenda.
\npara Every Referendum must be on a single issue of relevance to the JCR.  The result of a Referendum shall be binding on the General Meeting and the Executive Committee, in accordance with Articles \ref{Art:GeneralMeeting} and \ref{Art:TrusteeAccountability}, for two years.  During that two year period, the result of the Referendum may only be overturned by Referendum.
\npara A Referendum may be called by:
\begin{enumerate}
\item the General Meeting,
\item the Executive Committee, or
\item petition of thirty or more Ordinary Members, addressed to the Executive Committee via any Trustee, or
\item Members in accordance with the procedure set out in S.O. 9.
\end{enumerate}
\npara The Executive Committee shall have the right to dismiss any call for a Referendum if a Referendum has been held in the previous year on an issue that is substantively the same.  The Executive Committee must dismiss any call for a Referendum if the issue in question is not of relevance to the JCR.
\npara A Referendum must consist of one or more questions, which shall be determined by the Returning Officer.  After a Referendum, the Returning Officer shall produce a report on the conduct of the Referendum (or Referenda) and submit it to the College (via the relevant College Officer), the Executive Committee and the General Meeting within twenty Term-Time Days of the Referendum (or Referenda).  Provided they remain an Ordinary Member, they shall present it to the next OGM after submission.
\section{Organisation}
\npara If a Referendum is called in accordance with S.O. 5.3., the Returning Officer shall determine the precise wording of the question or questions, and the Polling Day. The Returning Officer shall give notice of the wording of any questions and answers and the Polling Day within seven Term-time Days of the Referendum having been called.  In exercising these powers, the Returning Officer must act in accordance with the requirements of S.O. 5.6.
\npara The Polling Day must be between seven and fourteen Term-time Days after the Referendum is called.  Notice of the wording of any questions and the Polling Day must be given at least five Term-time Days in advance of the Polling Day.
\npara If the Referendum concerns the JCR's continued affiliation to an External Organisation, then the question shall be as follows, where \textit{organisation} is replaced with the name of the External Organisation.  The available answers shall be ``Yes'', ``No'' and ``Abstain''.
``The JCR is currently affiliated to \textit{organisation}.  Should it continue to be affiliated''
\npara If the Referendum concerns ratification of a motion of no confidence, then the question shall be as follows, where \textit{name} is the name of the individual concerned and \textit{office} is the office that they hold.  The available answers shall be ``Ratify'', ``Reject'' and ``Abstain''.
``The JCR has no confidence in \textit{name} as \textit{office}.  \textit{Name} should be removed from office.''
\section{Campaigning}
\npara All Members have the right to campaign and produce publicity material in a Referendum in accordance with the Constitution and Standing Orders.
\npara Members must not spend money on campaigning (nor accept campaigning support of monetary value), except for any cost incurred by the printing of publicity materials, or with the express authority of the Returning Officer.
\npara The Returning Officer may issue further Guidance to ensure the free, fair and proper conduct of Referenda.
\subsection{Restrictions on Members}
\npara No Member shall take any action which might unfairly prejudice the outcome of the Referendum and the Returning Officer shall issue Guidance accordingly.  The general principle is that no Member should communicate their opinion by a means which could not reasonably be used by any Member who wished to do so.
\npara If necessary for the free, fair and proper conduct of the Referendum, the Returning Officer may require any Member to withdraw a statement that they have made or to apologise for an action that they have taken. 
\section{Voting}
\npara Only Ordinary Members may vote in Referenda and every Ordinary Member shall be entitled to one vote in a Referendum.
\npara Postal Votes shall not be accepted, except that Ordinary Members who are on a year abroad, or who are suspended from their studies, may vote in Referenda using a secure, secret online voting system chosen by the Returning Officer.
\npara No Member shall be obliged to use any or all of the votes at their disposal.
\npara The Returning Officer shall publicise the date and venue of Voting no fewer than five days before it is to take place. The Returning Officer shall post a guide to the voting procedure in a suitable place for the duration of Voting. The Returning Officer shall also notify via email members on a year abroad, or who are suspended from their studies, of the date of Voting, and of the online voting system to be used, no fewer than five days before Voting is to take place. The Returning Officer shall send such members a further email up to 24 hours before the start of Voting, providing the web address of the relevant online voting system and a guide to the voting procedure.
\npara Voting shall be open from 8.30am to 8.30pm on the Polling Day.
\npara The Returning Officer shall enlist scrutineers to supervise the Referenda on the day of Voting. Scrutineers may explain the completion of the ballot paper but shall not, whilst supervising Voting, advise or instruct any Member concerning the casting of their vote.
\npara Results of the Referenda shall be announced in Front Quad within three hours of the close of Voting, except in extraordinary circumstances.
\section{Complaints}
\npara Complaints concerning Referenda shall be dealt with in accordance with S.O. 10.
\npara In addition to any other remedies, the following remedies shall be available to provide effective remedy for a Complaint concerning a Referendum.
\begin{enumerate}
\item Require a Member to apologise for their campaigning activity.
\item Require a Member to withdraw or remove a statement or other publicity.
\item Annul the result of the Referendum.
\end{enumerate}

\npara Any remedy implemented must be proportionate for the purpose of ensuring that Referenda are free, fair and properly conducted.

