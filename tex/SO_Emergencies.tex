\chapter{Emergencies}
\textit{NB: This is an Agreed Standing Order}

\npara In the event that the JCR or any officer thereof is not reasonably able to satisfy an obligation or obligations imposed on them by the Constitution or Standing Orders due to unforeseen external circumstances, hereafter the ``Obligations'', the Executive Committee may negotiate with College an Emergency Waiver of the Obligations.
\npara The Emergency Waiver shall state which Obligations it covers, the nature of the unforeseen external circumstances, a start time in the future, and may (but is not required to) specify either an end time at a fixed date in the future, or some other condition at which the waiver expires. Where possible it shall contain Alternative Obligations, that ensure that the aims of the original Obligations are achieved as best as is reasonably possible.
\npara The Emergency Waiver shall have force if agreed to by the Executive Committee, College, and all Officers of the JCR subject to the Obligations, hereafter the Relevant Parties. In this case, while it remains in force, the Obligations are waived, and the Alternative Obligations shall have the same binding force as the original Obligations.
\npara The Alternative Obligations may be amended or removed entirely by agreement of the Relevant Parties.
\npara The Emergency Waiver may be rescinded by College at any time, giving reasonable notice to enable the JCR to resume compliance with the original Obligations, if in their opinion it is being misused.
\npara The details of the Emergency Waiver, or any amendment thereto, or the cancellation thereof, shall be communicated to all members of the JCR as soon as practicable, and if possible at least 24 hours before the time at which this change comes into force.
\npara Emergency Waivers affecting obligations in the Constitution shall only take force if the Constitution permits this.
\npara Emergency Waivers shall not affect obligations under this Standing Order.