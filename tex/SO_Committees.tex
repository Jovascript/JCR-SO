\chapter{Committees}
\textit{NB: This is an Internal Standing Order}
\section{General}
\npara The Committees of the JCR are:
\begin{enumerate}
	\item the JCR Committee,
	\item the Executive Committee,
	\item Standing Committees, and
	\item Sub-Committees.
\end{enumerate}
\npara All Committees shall operate in accordance with this Standing Order.
\npara The positions of Chair and Secretary shall never be held by the same individual.
\subsection{Chair of the Committee}
\npara Every Committee shall have a Chair.
The Chair shall be responsible for convening meetings on their own initiative (unless the Standing Orders or Founding Resolution require otherwise) and for coordinating the work of the Committee.
Except the Executive Committee, no Committee may meet outside of Full Term unless at least two-thirds of the Committee members agree otherwise.
\npara The requirement to have a Chair may be satisfied by having two co-Chairs, and all references to the Chair shall be construed accordingly.
\npara If the Chair is not determined by the Standing Orders or Founding Resolution, then the Committee members shall elect a Chair from within their own number.
\npara If the Chair is absent from a meeting and the Standing Orders do not provide otherwise, then the members of the Committee shall appoint one of their own number to act as Chair for the meeting.
\subsection{Secretary to the Committee}
\npara Every Committee shall have a Secretary.  The Secretary shall record business transacted (in the form of minutes) and maintain a file on the work of the Committee.
\npara If the Secretary is not determined by the Standing Orders or Founding Resolution, then the Committee members shall elect a Secretary from within their own number.
\npara If the Secretary is absent from a meeting and the Standing Orders do not provide otherwise, then the members of the Committee shall appoint one of their own number to act as Secretary for the meeting.
\npara The Secretary shall ensure that the minutes are published promptly to the Ordinary Members on the JCR Website, with the exception of any confidential information, by making them available to the Vice President or otherwise.
For the Executive Committee and the JCR Committee, this must be within five days of the meeting.
\npara The Secretary shall make other records of the Committee available to any Ordinary Member on request, unless such disclosure would be prejudicial to the interests of the JCR.
\npara Liaising with the Chair, the Secretary shall compile the Agenda.
All members of the Committee shall have the right to contribute to the Agenda for each meeting, by submission in advance or at the start of the meeting.
\subsection{Further procedure}
\npara If a member of the Committee wishes, a vote on any issue shall be taken, with the decision made by simple majority.
There is one vote per position on the Committee.
The Chair has casting vote. No proxy votes or absentee votes will be accepted.
\npara Any Member may make a proposal to the Committee.
\npara Any Member may be invited to attend and speak at a Committee meeting by resolution of the Committee, or on the initiative of the Chair.  If there is good reason to believe that it would be in the interests of the JCR, any other person may be invited to attend and speak at a Committee meeting by resolution of the Committee, or on the initiative of the Chair.
Only members of the Committee may vote.
\section{The Executive Committee}
\npara All members of the Executive Committee shall have access to the JCR Store Room, but only the President and Entertainment Representatives shall have access to the area used to store alcoholic beverages.
\subsection{Procedure}
\npara The President shall act as the Chair to the Executive Committee.
If they are absent or otherwise unable, the Chair shall be taken by the Treasurer, Access and Equal Opportunities Representative, Academic Affairs Representative, either Welfare Officer or either Entertainment Representative in that order.
\npara The Chair shall convene meetings of the Executive Committee weekly in Full Term and more frequently at their discretion.
\npara The Agenda shall be compiled by the Committee Chair at the beginning of the meeting.
All members of the Executive Committee have the right to contribute to the Agenda.
\npara The Vice President shall act as Secretary to the Executive Committee.
If they are absent or otherwise unable, the Academic Affairs Representative, Access and Equal Opportunities Representative, Treasurer, either Welfare Officer or either Entertainment Representative shall take over secretarial duties, in that order.
\npara A report of all business transacted by the Executive Committee since the last committee meeting shall be made to the JCR Committee at the subsequent JCR Committee meeting in the section of the Meeting designated "Executive Committee Business" (S.O. \ref{transacted_business}.\ref{exec_business}).
\npara The General Meeting shall have the power to review decisions of the Executive Committee, and may issue Directions in accordance with Article \ref{Art:TrusteeAccountability}.
\subsection{Responsibilities}
\npara The general control and management of the administration of the JCR, in accordance with the Constitution and Standing Orders.
\npara The day-to-day administration of the JCR under the supervision and coordination of the President
\npara The day-to-day administration of the JCR under the supervision and coordination of the President
\npara Providing for the welfare, entertainment, facilities and other business of the JCR.
\npara Assisting the President and other members of the Executive Committee in the execution of their duties and deputising as appropriate.
\npara Implementing resolutions of the General Meeting and the JCR Committee to the best of their ability.
\npara Arranging a further, group, handover session between the outgoing and incoming Executive Committees prior to the latter taking up office. This shall supplement and not replace the personal handover required in S.O. 2.3.2.1.v.
\npara Supervising the work of the JCR Committee and all Officers.
\npara Supervising the work of any Officer for which they are the Designated Officer.
\npara The carrying out by individuals of such further duties that the Executive Committee determines.
\subsection{Powers}
\npara The Executive Committee shall be responsible for the general control and management of the administration of the JCR and shall have all lawful powers necessary or convenient for that purpose, subject to the restrictions of the Constitution and Standing Orders, including the restriction of any powers to Members.
\npara The Executive Committee shall have all of the powers laid out in Article \ref{Art:TrusteePowers}.
But, in accordance with that Article, the Executive Committee shall not take any action or make any decision unless it is satisfied that any liabilities which are likely to result will be met out of the JCR's assets or covered by suitable insurance.
The Executive Committee may determine general criteria for sufficient satisfaction annually or more frequently.
\section{The JCR Committee}
\subsection{Procedure}
\npara The JCR Committee shall meet at least once a fortnight during Full Term, and additionally at the discretion of the Committee Chair.
\npara The JCR Committee shall be convened and chaired by the Committee Chair.
\npara \label{transacted_business}The following business should be transacted:
\begin{enumerate}
	\item Summary of business transacted at the previous Committee Meeting
	\item Matters arising
	\item Notices
	\item \label{exec_business}Executive Committee Business (see S.O. 2.2.2.5.)
	\item Committee Business
	\item Relevant business from College committees
	\item Designating Representatives for upcoming College committees
	\item Sub-Committee (see S.O. 2.4.2.1) and Standing Committee (see S.O. 2.5.2.1) Business
	\item Any other business
\end{enumerate}
\npara The Executive Committee shall present a report of recent business as in S.O. 2.2.2.5, omitting the details of any confidential business.
Business designated as confidential by the Executive Committee shall be mentioned only briefly, although an item's confidentiality may be queried by any member of the JCR Committee.
\npara All JCR representatives on College committees and sub-committees shall deliver a short verbal report on business conducted there to the JCR Committee.
The JCR Committee shall have the right to ask short factual questions, and to request that the short verbal report be delivered to the OGM.
\npara The Chair shall list all upcoming College committee meetings on which the JCR has representation.
If the JCR place is not assigned to a specific JCR Committee member or Sub-Committee or the designated individual is unable to attend, the Committee shall select an individual to represent the JCR.
If a vacancy arises and the JCR Committee is not due to meet before the College committee meeting, this shall be the responsibility of the President.
All JCR representatives on College committees and sub-committees shall be selected from among the members of the JCR Committee.
\npara Any Member may observe silently at a JCR Committee meeting, except in the case of matters requiring confidentiality, which may be reserved for members of the Committee.
\npara It shall be in the power of the General Meeting to review, and if necessary overturn, a JCR Committee decision.
\npara For each Non-Executive member of the JCR Committee, there shall be one or more Designated Officers, who are members of the Executive Committee and are responsible for supervising the Officer in carrying out their duties.
\subsection{Responsibilities}
\npara Each member of the JCR Committee shall undertake the following responsibilities.
\begin{enumerate}
	\item Attending all General Meetings and JCR Committee meetings, as well as Sub-Committee and Standing Committee meetings pertaining to their position. The relevant Secretary must be notified of any expected absence before the start of the meeting.
	\item Acting in accordance with the Constitution and Standing Orders at all times whilst executing their office.
	\item As required by the President, attending meetings of Affiliated Organisations, voting on behalf of the JCR, and reporting back to the JCR where appropriate.
	\item Producing a written report at the end of each term, summarising all work undertaken and highlighting improvements to be made next term, to be emailed to the JCR by the President by Friday of 10th Week.
	\item Arranging a handover meeting with their successor where relevant information and any archives pertaining to the position shall be handed over and a discussion of ongoing business shall take place, as well as writing a handover document including at least a list of duties and regular tasks as well as summaries of the state of current projects.
	\item Maintaining the condition of any College or JCR property which their position places in their charge.
	\item Updating the parts of the JCR website relevant to their position or making information available to the Vice President so they may do so.
	\item Carrying out their duties in accordance with the decisions of the Executive Committee.
	\item Whilst acting \textit{ex officio}, acting in accordance with the resolutions of the General Meeting, except in extraordinary circumstances, in which case they must justify their conduct to the next OGM.
\end{enumerate}
\subsection{Powers}
\npara The power to request information from the Executive Committee in accordance with S.O. 2.3.1.4.
\npara Such further powers as are necessary to fulfil their responsibilities in accordance with the Constitution and Standing Orders.
\subsection{Associate Members and the Committee Chair}
\npara The JCR Committee may have amongst its number Associate Members, appointed by a motion at a General Meeting and removable thereby or determined by Standing Order. Associate Members are expected to attend, and may speak at, meetings of the JCR Committee, but they may not vote in them. They may also receive the minutes of those meetings. Primarily present in an informative and advisory capacity, they do not have any of the general responsibilities towards the JCR described below, bar attendance at meetings, and any decision of the JCR Committee or General Meeting cannot be binding upon them.  Associate Members must be members of the College.
\npara Each Term, the JCR Committee shall elect a Committee Chair to serve for the duration of the following Term, from amongst its Non-Executive members.  In the event of a vacancy, a further election may be held.
\npara The Chair of any Standing Committee or Sub-Committee shall be an Associate Member, unless they are a member of the JCR Committee.
\section{Sub-Committees}
\subsection{General}
\npara In accordance with Article \ref{Art:Sub-Committees}, the Executive Committee may establish Sub-Committees to support its work; any such Sub-Committee shall have a clearly defined mandate lasting not more than one year.  The Founding Resolution shall clearly specify the mandate of the Sub-Committee, the length of time for which it shall operate and any powers delegated to it in accordance with Article \ref{Art:Delegation}.  The Executive Committee shall present the Founding Resolution to the next OGM.
\npara In accordance with Article \ref{Art:Sub-Committees}, the General Meeting may establish permanent Sub-Committees by Standing Order.  The General Meeting shall do this by amending S.O. 2.6.  The Executive Committee must consent to the delegation of any powers to such a Sub-Committee, and retain the right to amend or withdraw such a delegation in accordance with Article \ref{Art:Delegation}.
\npara All members of Sub-Committees are required to attend all meetings and fulfil all responsibilities of membership.
\subsection{Responsibilities}
\npara The Sub-Committee shall be directly responsible to the Executive Committee and shall provide full and timely reports of its acts and proceedings to the Executive Committee, via the Chair.  The Executive Committee shall provide regular reports to the JCR Committee and the General Meeting on the work of its Sub-Committees.
\npara The Sub-Committee shall carry out its mandate to the best of its ability, in accordance with the decisions of the General Meeting and the Executive Committee.
\subsection{Powers}
\npara The Sub-Committee shall have such powers as are necessary to operate in accordance with the Standing Orders and such powers as are delegated to it in accordance with Article \ref{Art:Delegation}.
All Sub-Committee Powers and Responsibilities stem from those of the Committee Chair.
\section{Standing Committees}
\subsection{General}
\npara The General Meeting shall have the power to appoint a Standing Committee with a specific mandate to be carried out within a predetermined period of time of no longer than one year. 
\npara The Standing Committee shall carry out its mandate to the best of its ability, in accordance with the decisions of the General Meeting.
\npara Membership is to be decided by appointment in the Founding Resolution.  The Chair (or if there are co-Chairs, one of the co-Chairs) must be an Ordinary Member.
\npara The Chair shall give a report of business transacted by the Standing Committee at meetings of the JCR Committee during the section of the meeting designated "Sub-Committee and Standing Committee Business" (S.O. 2.3.1.3.viii).
\subsection{Responsibilities}
\npara The Standing Committee shall be directly responsible to the General Meeting and the Chair shall provide regular reports to the General Meeting and to the JCR Committee.
\npara The Chair shall present a written report of all findings and proposals to the General Meeting after the fulfilment of the mandate of the Committee.
\subsection{Powers}
\npara The Standing Committee shall have such powers as are necessary to fulfil its mandate, arising from the powers of the General Meeting in accordance with S.O. 2.5.4.2.
\npara The Standing Committee exists primarily to support the General Meeting and shall have only the powers necessary for that purpose, unless the Executive Committee delegates powers in accordance with Article \ref{Art:Delegation}.
\npara The Standing Committee shall act only in order to fulfil its mandate and shall cease to exist once its purpose has been fulfilled.
\section{Permanent Sub-Committees of the Executive Committee}
\npara The following permanent Sub-Committees of the Executive Committee shall be established, in accordance with Article \ref{Art:Sub-Committees}.
\npara Where relevant, volunteering Members approved by the Chair hold membership of the Sub-Committee for one year from the date of first attendance at the Committee, unless this is curtailed in accordance with S.O. 2.7.
\subsection{Access Sub-Committee}
\npara Chaired by the Access and Equal Opportunities Representative, this Sub-Committee shall be responsible for assisting the Access and Equal Opportunities Representative in encouraging applications to the College and the University from institutions and groups with little experience of doing so and is a forum for the discussion of both current and future access initiatives.
\npara The Access Sub-Committee shall meet at least once a term.
\npara The members of the Sub-Committee shall be the Access and Equal Opportunities Representative, the Vice President, at least one additional member of the Equality Sub-Committee co-opted by the Chair, the Publications and Social Media Officer and any volunteering Members approved by the Chair.
\subsection{Equality Sub-Committee}
\npara Chaired by the Access and Equal Opportunities Representative, this Sub-Committee shall be responsible for promoting equality within the College and the University and acting to prevent discrimination or prejudice of any kind, in accordance with Article \ref{Art:Equality} of the Constitution. To this end, the Sub-Committee shall ensure that, as far as possible, there is a Merton presence at all external committees and campaigns convened by Affiliated Organisations for the consideration and advancement of issues relating to the issues outlined above.
\npara The Equality Sub-Committee shall always meet in advance of College Equality Committee meetings and at least once a term.
\npara The Equality Sub-Committee shall agree on two representatives to send to the termly College Equality Committee meeting, based on how relevant the various agenda items are to their role.
\npara The Equality Sub-Committee shall organise and support College in organising events promoting awareness of equality matters.
\npara Subject to the availability of spaces and SO 3.9.2.xiii., members of the Equality Sub-Committee are encouraged to be Peer Support trained.
\npara The members of the Sub-Committee shall be the Gender Equality Representative, the Access and Equal Opportunities Representative, the International Students’ Representative, the BME Representative, the Disabled Students’ Representative, the Social Backgrounds Representative and the LGBTQIA+ Representative.
\subsection{Welfare Sub-Committee}
\npara Co-chaired by the Welfare Officers, this shall be a forum for the exchange of ideas for the improvement of the welfare and safety of Undergraduates.
\npara The Welfare Sub-Committee shall meet at least once in each Term.
\npara The members of the Sub-Committee shall be the two Welfare Officers, the two Welfare Helpers, the Access and Equal Opportunities Representative, the Gender Equality Representative, the International Students’ Representative, the LGBTQIA+ Representative, the Disabled Students’ Representative, the BME Representative, the Social Backgrounds Representative and any volunteering Members approved by the Chair.
\npara Subject to the availability of spaces and SO 3.9.2.xiii., members of the Welfare Sub-Committee are encouraged to be Peer Support trained
\section{Resignation and removal}
\npara Any Office-Holder may resign at any time in accordance with Articles \ref{Art:Dismissal} and \ref{Art:TrusteeAccountability}.
\subsection{Motions of no confidence}
\npara The General Meeting may pass a motion of no confidence in any Office-Holder by Special Resolution.  The General Meeting shall not pass such a motion unless:
\begin{enumerate}
	\item the Office-Holder concerned has been given notice in writing at least 96 hours in advance of the meeting that the motion is to be proposed, specifying the reasons for the proposed removal from office;
	\item the Office-Holder has been afforded a reasonable opportunity of being heard by or, at the option of the Office-Holder, of making written representations to the General Meeting; and
	\item the Ordinary Members have been given notice at least 48 hours in advance of the meeting that the motion is to be proposed.
\end{enumerate}
\npara If passed by the General Meeting, a motion of no confidence must be ratified before it takes effect.  Immediately a motion of no confidence in an Office-Holder is ratified, they shall cease to hold the designated office and any other offices that they hold by virtue of that office.
\npara Ratification of a motion of no confidence in a Trustee to remove them from the Executive Committee shall be by Referendum, in accordance with S.O.5.  The passing of such a motion of no confidence by the General Meeting shall be considered a call for a Referendum.
\npara Ratification of any other motion of no confidence shall be by the Chair of the General Meeting approving the Minutes of the General Meeting at which the motion passed as a true and fair record.
\subsection{Executive Fiat}
\npara The Executive Committee may suspend any Office-Holder for up to one week of Full Term in any given Term and such further periods out of Full Term as may be necessary.  The Executive Committee shall not suspend an Office-Holder unless:
\begin{enumerate}
	\item the Office-Holder concerned has been given notice in writing at least 96 hours in advance of the meeting that the suspension is to be proposed, specifying the reasons for the proposed removal from office;
	\item the Office-Holder has been afforded a reasonable opportunity of being heard by or, at the option of the Office-Holder, of making written representations to the Executive Committee; and
	\item the decision is taken at a meeting and the Trustees have been given notice at least 48 hours in advance of the meeting that the suspension is to be proposed.
\end{enumerate}
\npara Immediately the Executive Committee gives notice to an Office-Holder that they have been suspended, the Office-Holder shall be relieved of the relevant duties and responsibilities and shall be prohibited from purporting to act in that office, subject to S.O. 2.7.3.3.
\npara If a Trustee is suspended, they retain the right to attend, speak and vote at meetings of the Executive Committee and its Sub-Committees, subject to the Constitution and Standing Orders.
\npara For the avoidance of doubt, a Trustee has a Conflict of Interest in deciding on their own suspension.
\npara Where an Office-Holder has a Designated Officer, the Executive Committee shall not suspend that Office-Holder without the approval of the Designated Officer, unless the Executive Committee is convinced that it is necessary to do so.