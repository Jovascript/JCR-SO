\chapter{Complaints}
\textit{NB: This is an Agreed Standing Order.}
\npara In accordance with the Education Act 1994, there shall be a complaints procedure to deal with complaints of the nature described in Article \ref{Art:Complaints} of the Constitution.
\npara Any Undergraduate or group of Undergraduates, hereinafter the Complainants, shall have the right to submit a Complaint regarding any of the matters referred to in Article \ref{Art:Complaints}, in accordance with this S.O.
\npara For any Complaint, the relevant process must be identified, in accordance with S.O. 10.5, and the stages outlined in this S.O. must be followed in order.  Only if the Complaint is not dealt with to the satisfaction of all relevant parties shall any unsatisfied party have the right to appeal to the next stage.
\npara Complaints must be dealt with fairly and promptly, and if a Complaint is upheld, there must be an effective remedy.  The Executive Committee is responsible for overseeing the procedure, ensuring that robust processes are in place, and ensuring the implementation of an effective remedy where necessary (unless such a remedy is implemented by the College).
\section{Processes}
\npara Before making a Complaint, any Undergraduate or group of Undergraduates may seek informal resolution of the matter by contacting the President or, if the matter relates to the President in a way that would make this inappropriate, any other Trustee.  No one shall be obliged to seek informal resolution before making a Complaint.
\npara Any Complaint which relates to Harassment shall be dealt with in accordance with the College's harassment procedure.
\npara Any Complaint which relates to an Election or Referendum shall be dealt with in accordance with the procedure in S.O. 10.7.
\npara Any other Complaint shall be dealt with in accordance with the procedure in S.O.10.6.
\npara All parties to the the Complaint shall maintain appropriate confidentiality at all times, in accordance with S.O. 10.8.
\npara The steps taken by the person or body investigating the Complaint may include (as appropriate):
\begin{enumerate}
\item meeting separately with the Complainants and any person who is a subject of the Complaint;
\item interviewing individuals identified by any parties to the Complaint as having relevant evidence;
\item speaking to other relevant people on a confidential basis;
\item considering whether to request the taking of immediate interim action;
obtaining further relevant information.
\end{enumerate}
\section{JCR Complaints}
\subsection{Initial Stage}
\npara In the first instance, the Complaint shall be submitted by email to a Trustee, hereinafter referred to as the Complaint Trustee, using the email address listed on the JCR Website and including the words ``Formal Complaint'' in the subject line.  Unless there is good reason otherwise, the Complaint Trustee should be the Vice President.
\npara The aim shall always be to conclude the Complaint within no more than ten working days during term time, and no more than twenty working days outside of term time, of the Complaint being submitted. If this is not possible then all parties to the Complaint shall be informed of the likely time frame and shall be notified of the progress no less than once a fortnight, by either the Complaint Trustee or the Complaint Panel.
\npara If the Complaint Trustee determines that the Complaint is not well founded, then they may, giving reasons, dismiss the Complaint without appointing the Complaint Panel provided in S.O. 10.6.1.4. The Complainants may appeal against such a decision in accordance with S.O. 10.6.2.
\npara Consulting with the Executive Committee as appropriate, the Complaint Trustee shall appoint a Complaint Panel, of between one and three persons, to investigate the Complaint.  The Complaint Panel shall be a Sub-Committee of the Executive Committee.  The Complaint Trustee may be a member of the Complaint Panel and at least one member of the Complaint Panel must be an Ordinary Member.  Every member of the Complaint Panel must agree in writing to adhere to the confidentiality requirements in S.O. 10.8.
\npara The Complaint Panel shall take such steps as they think necessary or appropriate to reach a decision on the outcome of the Complaint.
\npara If the Complaint is made against a person or persons, the Complaint Panel shall inform them of the details of the allegations and provide them with the opportunity to provide any relevant evidence and make written and oral representations in their defence. Representations may not be made by another person on their behalf, even if that person is legally qualified.
\npara In investigating the Complaint, the Complaint Panel may determine that a task should be undertaken by one or more members of the panel.
\npara The Complaint Panel shall inform the Complaint Trustee, the Executive Committee (via the Chair), the Complainants and any person who is a subject of the Complaint by email of:
\begin{enumerate}
\item the conclusions they have reached having reviewed the evidence and whether the Complaint
\begin{enumerate}
\item is upheld (in whole or in part), or
\item is rejected (including, if appropriate, that the Complaint is unfounded and not made in good faith);
\end{enumerate}
\item any action that they recommend by way of effective remedy;
\item the reasons for any such recommendation.
\end{enumerate}
\npara If the Complaint is upheld, the Executive Committee shall implement an effective remedy, having due regard to the conclusions and recommendations of the Complaint Panel.
\subsection{Appeal stage}
\npara If, having completed the initial stage, the Complainants or any person against whom the Complaint was made does not accept the outcome of the Complaint, they may refer the issue to the Senior Treasurer of the JCR, to be dealt with in accordance with the Bylaws of the College.  The Complaint Trustee must be informed by email if the issue is referred to the College.
\npara A referral in accordance with S.O. 10.6.2.1. may only be made within five days of receiving the notification specified in S.O. 10.6.1.8. (or, if no part of the five days would fall within Full Term, within ten days of receiving the notification).
\npara In accordance with the relevant procedures, the College shall have the power to overturn, uphold or alter the decision of the Complaint Panel at the initial stage.
\npara If the Complaint is upheld, the Executive Committee and the College shall implement an effective remedy.
\section{Election and Referendum Complaints}
\npara No Complaint may be submitted under this procedure more than 96 hours after the result of the relevant Election or Referendum is announced, unless new evidence becomes available.  In any case, no Complaint may be submitted under this procedure more than two weeks after the result of the relevant Election or Referendum is announced.
\subsection{Initial stage}
\npara In the first instance, the Complaint shall be submitted by email to the Returning Officer or, if the Complaint concerns the Returning Officer, to a suitable Trustee.  The Complainants must use the email address listed on the JCR Website and include the words ``Formal Complaint'' in the subject line.  The person to whom the Complaint is submitted is hereinafter referred to as the Complaint Officer.
\npara The Complaint Officer shall undertake an initial investigation of the Complaint, to determine the relevant facts.  The Complaint Officer shall have the power to require relevant Members to meet with them and to require a Member to give a prompt answer to any reasonable question.
\npara If the Complaint Officer determines that the Complaint is not well founded, then they may, giving reasons, dismiss the Complaint.  The Complainants may appeal against such a decision in accordance with S.O. 10.7.3.
\npara Within 24 hours of receiving the Complaint, the Complaint Officer must take one of the following three actions.
\begin{enumerate}
\item Decide whether or not to uphold the Complaint and, if the Complaint is upheld, determine an effective remedy.
\item Extend their investigation, but not beyond 48 hours after they received the Complaint.
\item Direct the Executive Committee to appoint a Complaint Panel of two or three suitably independent persons to investigate the Complaint.  The Complaint Panel shall be a Sub-Committee of the Executive Committee.  The Complaint Officer may be a member of the Complaint Panel and at least one member of the Complaint Panel must be an Ordinary Member.  Every member of the Complaint Panel must agree in writing to adhere to the confidentiality requirements in S.O. 10.8.  The Complaint Panel shall aim to conclude within five days.  If this is not possible then all parties to the Complaint shall be informed of the likely time frame and shall be notified of the progress no less than once a week, by either the Complaint Officer or the Complaint Panel.
\end{enumerate}
\npara In their investigation, the Complaint Officer or the Complaint Panel shall take such steps as they think necessary or appropriate to reach a decision on the outcome of the Complaint.
\npara If the Complaint Officer has extended their investigation as permitted by S.O. 10.7.2.4(ii), the Complaint Officer must either reach a determination on the Complaint and any remedy, or direct the Executive Committee to appoint a Complaint Panel as described above within 48 hours of receiving the Complaint.
\npara If the Complaint is made against a person or persons, the Complaint Officer or the Complaint Panel shall inform them of the details of the allegations and provide them with the opportunity to provide any relevant evidence and make written and oral representations in their defence. Representations may not be made by another person on their behalf, even if that person is legally qualified.
\npara In investigating the Complaint, a Complaint Panel may determine that a task should be undertaken by one or more members of the panel.
\npara The Complaint Officer or the Complaint Panel shall inform the Returning Officer, the Executive Committee (via the Chair), the Complainants and any person who is a subject of the Complaint by email of:
\begin{enumerate}
\item the conclusions they have reached having reviewed the evidence and whether the Complaint
\begin{enumerate}
\item is upheld (in whole or in part), or
\item is rejected (including, if appropriate, that the Complaint is unfounded and not made in good faith);
\end{enumerate}
\item any action that they recommend by way of effective remedy;
\item the reasons for any such recommendation.
\end{enumerate}
\npara If the Complaint is upheld, the Complaint Officer or the Complaint Panel shall implement an effective remedy, in accordance with their conclusions and recommendations, assisted by the Executive Committee and Returning Officer as appropriate.  For a Complaint regarding an Election, the remedies available shall include those listed in S.O. 4. and for a Complaint regarding a Referendum, the remedies available shall include those listed in S.O. 5.
\subsection{Second Internal Stage}
\npara If the initial stage was conducted in a manner which was procedurally irregular or there is new evidence which could not previously have been considered (but not for any other reason), any party to the Complaint may appeal to the Executive Committee via any Trustee.  The Complaint Officer must be informed if such an appeal is made.
\npara An appeal from a decision of the Complaint Officer may only be made within 24 hours of receiving the notification specified in S.O. 10.7.2.9.  An appeal from a decision of a Complaint Panel may only be made within five days of receiving the notification specified in S.O. 10.7.2.9.  If the Executive Committee determines that the appeal is not well founded, then it may dismiss the Complaint. 
\npara The Executive Committee shall take such steps as they think necessary or appropriate to establish whether there are grounds for changing the determination made at the initial stage.  The Executive Committee shall aim to conclude within five days.  If this is not possible then all parties to the Complaint shall be informed of the likely time frame and shall be notified of the progress no less than once a week.
\npara The Executive Committee shall inform the Returning Officer, the Senior Treasurer of the JCR, the Complainants and any person who is a subject of the Complaint by email of:
\begin{enumerate}
\item the conclusions they have reached having reviewed the evidence and whether the Complaint
\begin{enumerate}
\item is upheld (in whole or in part), or
\item is rejected (including, if appropriate, that the Complaint is unfounded and not made in good faith);
\end{enumerate}
\item any action that they recommend by way of effective remedy;
the reasons for any such recommendation.
\end{enumerate}
\npara If the Complaint is upheld, the Returning Officer and Executive Committee (as appropriate) shall implement an effective remedy, having due regard to the conclusions and recommendations given in accordance with S.O. 10.7.3.4.  For a Complaint regarding an Election, the remedies available shall include those listed in S.O. 4. and for a Complaint regarding a Referendum, the remedies available shall include those listed in S.O. 5.
\subsection{Appeal to the College}
\npara If, having completed initial stage and, if applicable, the second internal stage, the Complainants or any person against whom the Complaint was made does not accept the outcome of the Complaint, they may refer the issue to the Senior Treasurer of the JCR, to be dealt with in accordance with the Bylaws of the College.  The Returning Officer and the Executive Committee must be informed by email if the issue is referred to the College.
\npara If appealing from a decision of the Complaint Officer, such a referral may only be made within 24 hours of receiving the notification specified in S.O. 10.7.2.9.  In all other cases, a referral may only be made within five days of receiving the notification specified in S.O. 10.7.2.9. or S.O. 10.7.3.4. as applicable (or, if no part of the five days would fall within Full Term, within ten days of receiving the notification).
\npara In accordance with the relevant procedures, the College shall have the power to overturn, uphold or alter the decision of the Complaint Panel at the initial stage.
\npara If the Complaint is upheld, the Executive Committee and the College shall implement an effective remedy.  For a Complaint regarding an Election, the remedies available shall include those listed in S.O. 4. and for a Complaint regarding a Referendum, the remedies available shall include those listed in S.O. 5.
\section{Confidentiality Requirements}
\npara All details of the administration and determination of a Complaint, and of the Complaint itself, shall be treated as confidential unless the individual or body presiding over the Complaint determines otherwise.
\npara In judging the appropriate level of confidentiality, a balance must be struck between protecting the privacy of individuals and a level of transparency that maintains trust in the complaints process.
\npara The Executive Committee shall ensure that a general summary of each Complaint received and whether it was upheld is published to the College and the Ordinary Members.  This summary must not contain confidential information, in accordance with S.O. 10.8.1.
\npara The identity of the Complainants should not normally be revealed outside of the process of investigating and determining the Complaint, although it may be necessary or appropriate to do so in some circumstances.





