\chapter{Room Ballot}\label{App:RoomBallot}

\section{Room Ballot}

\appnpara{The overall order of choice on the Final Room Ballot shall consist of Fourth Year Ballot at the top (with respect to point 2.5 below), followed by the Third Year Ballot, with the Second Year Ballot at the bottom.}
\npara A ballot of all incoming fourth years, with the exception of Law Course II students and Modern Linguists, shall be drawn no later than Sunday of 6th Week of Hilary Term. Incoming fourth year Modern Linguists and Law Course II students may undertake to join the fourth year ballot by response to an e-mail sent by the Vice President in Hilary Term. These students will be able to ballot with other incoming fourth years in that ballot. Failure to respond to the Vice President's email will be treated as a desire to stay in their usual position. This does not prevent a fourth year who stays in the third year ballot from choosing a remaining room in a fourth year house, if one is left when they come to choose a room.

\npara \label{AppP:RoomIncomingThird} The order of choice in the Room Ballot for incoming third and fourth year Law Course II students and Modern Linguists and incoming fourth year Classicists shall be determined in the following order:
\begin{enumerate}
\item The second year ballot from the previous year shall be inverted.
\item Fourth year Modern Linguists shall be reinserted into the ballot at the position they would have occupied in their third year had they been in residence. \\ \\
For example, if last year's third year ballot was as follows:\\ \\
1. Tim Smith\\
William Pogge (removed linguist)\\
2. Francesca Rawls\\
3. Patty Green\\
Eleanor Wild (removed linguist)\\
Nicholas Poole (removed linguist)\\
Then William Pogge shall be given second place, Eleanor Wild fifth place, and Nicholas Poole sixth place. They shall then be put back into this year's ballot in the following manner:\\
1.	Sam Frost (normal third year)\\
2.	Emily Iversen (normal third year)\\
2a. William Pogge (returning linguist)\\
3.	Jack Miles (normal third year)\\
4.	Ann Law (normal third year)\\
5.	Vince Mint (normal third year)\\
5a. Eleanor Wild (returning linguist)\\
6	Phil Saundry (normal third year)\\
6a. Nicholas Poole (returning linguist)\\ \\ \\
Finally, their position relative to the other third year at that position shall be determined by the toss of a coin. For example, if Tails is taken to mean that the returning linguist takes the normal third year's position and all returning linguists come up Tails, then the room ballot should look like this:\\ \\
1. Sam Frost (normal third year)\\
2. William Pogge (returning linguist)\\
3. Emily Iverson (normal third year)\\
4. Jack Miles (normal third year)\\
5. Ann Law (normal third year)\\
6. Eleanor Wild (returning linguist)\\
7. Vince Mint (normal third year)\\
8. Nicholas Poole (returning linguist)\\
9. Phil Saundry (normal third year)\\
\item Law Course II students shall be reinserted into the ballot at the position they would have occupied in their third year (but for point 1.3.5.) in a similar manner as above, their position relative to the other third year at that position being determined by the toss of a coin.
\item Undergraduates who have been out of residence for the preceding year and who are returning for the academic year to which the ballot relates as third years shall be inserted into the ballot at their third year position.
\item \label{AppP:RoomLinguist} Third year Modern Linguists and Law Course II students who are spending their third year out of residence shall be removed from the third year ballot.
\item Undergraduates who have been out of residence for the preceding year and who are returning for the academic year to which the Ballot relates as Second Years shall be inserted into the Ballot at their second year position.
\item Any undergraduate (such as EUMEL or Russian modern linguist) who takes a year abroad in their second year will be treated as follows. They will be allowed to ballot with their friends in the first year, as if they were staying in College for their second year. Their place in the third year room ballot will therefore be determined by the inversion of the second year ballot like regular undergraduates. For their fourth year, they will be included in the fourth year room ballot referred to in point 1.2.
\item Undergraduates who have suspended their studies shall be inserted into the ballot which relates to their academic year, or they may undertake to resume their position in the ballot which relates to their matriculation year by response to an email sent by the Vice President in Hilary Term. In the case of the latter, for the following year, they shall be inserted into the ballot which relates to their academic year. Insertion into their academic year will be by random number generator or balloting as part of a group if appropriate to that year and before the usual deadlines.
\item To insert by random number generator, the process shall be carried out as follows:
\begin{enumerate}
\item Identify the total number of students currently in the ballot (excluding any students due to be inserted into the ballot), N,
\item Choose a random number, k, between 1 and N + 1,
\item Insert the student into the kth position, moving lower students down.
\end{enumerate}
\item The insertion of students by random number generator shall happen before the insertion of modern linguists, classicists and Law course II students. In the case of inserting more than one student, this process outlined in 1.3.10 shall be repeated with N increasing each time.
\item For the Room Ballot occuring in 2023 only, incoming fourth year Classicists may undertake to join the fourth year ballot by response to an e-mail sent by the Vice President in Hilary Term. These students will be able to ballot with other incoming fourth years in that ballot, except if they are placed in the lower half of the fourth year ballot, they shall be moved into the bottom of the upper half of the Ballot, maintaining their ordering with other incoming fourth year Classicists. Failure to respond
to the Vice President’s email will be treated as a desire to be reinserted  into the third year ballot at the position they would have occupied in their third year (had they not been moved to the bottom), as with Modern Linguists. This clause shall expire at the end of the 2022-23 Academic Year.
\end{enumerate}

\appnpara Incoming second years may enter the room ballot automatically as individuals or voluntarily in groups of up to six members.
\appnpara The Vice President shall solicit for group entries to the Second Year Ballot, the deadline for which shall be Wednesday 6th Week of Hilary Term at 6pm.
\appnpara On Sunday of 7th Week, Hilary Term, group and individual entries to the Second Year Ballot will be placed on the ballot in the order of their being drawn out of a hat which contains one piece of paper for each individual or group. On each occasion that a group entry is drawn out, a separate paper draw will take place to determine the order in which the individual members of that group shall appear on the ballot.
\appnpara The Final Room Ballot shall be emailed to all members within 24 hours of the drawing of the Second Year Ballot.

\section{Room Choosing}
\appnpara A list of the available rooms and floor plans, and photos where they exist, thereof shall be emailed to all members by Friday of 8th Week, Hilary Term, subject to being made available by the Accommodation Manager or Estates Bursary.
\appnpara Viewing of rooms shall take place as available during 1st and 2nd Week of Trinity Term.
\appnpara Room choosing for fourth and third years shall take place in 3rd Week of Trinity Term.
\appnpara Room choosing for second years shall take place in 5th Week of Trinity Term.
\appnpara Fourth years, except fourth year classicists, lawyers and modern linguists, may only choose rooms in the houses which the JCR has designated as fourth year houses. Fourth year classicists, lawyers and modern linguists who have joined the fourth year ballot are also limited to choosing from the fourth year houses.
\appnpara A member may give permission for another member to choose a room for them, subject to the Vice President being notified of such permission at least 24 hours prior to room choosing.
\appnpara Members intending not to live in College accommodation must notify the Vice President of this at least 24 hours prior to room choosing.
\appnpara In the event that a member does not attend room choosing, and they have not informed the Vice President that another member will be choosing a room for them or that they are not intending to live in College accommodation, the Vice President shall choose what they consider to be the most suitable room on behalf of that member.
\appnpara The JCR President-elect and Vice President-elect shall choose whether or not they will reside in their designated rooms before the room choosing date applicable to them. Should this not be possible, they will be given the option to reside in the designated rooms after their room choosing. These rooms will be made available to the JCR as in \ref{AppP:RoomAvailable}.
\appnpara \label{AppP:RoomAvailable} In the event of a room becoming available after room choosing, this will be advertised to the JCR and the room will be made available to the person highest on the room ballot who wishes to move into that room.
